%! Author = danjane
%! Date = 17.09.22

\documentclass[10pt]{article}

\usepackage{amsmath}
\usepackage{cmbright}

\begin{document}

\section{Project proposal}

Bucheton and Soulé have described the act of teaching as a multi-agenda game of postures requiring good preparation and excellent micro-decisions \cite{BS09}. In their model, they identify the crucial roles of the teacher in controlling the cadence, the atmosphere, the scaffolding and the relationships during the class, and how each supports the learning objective. And yet many of the teacher's most time-consuming tasks do not take place in the classroom: good preparation and strong follow-ups (for example auto-reflection, marking and parent-teacher interactions) work to support and complement the pillars identified in the student learning process.

Many of these tasks are ripe for automation. As Sweigart writes, ``many people spend hours clicking and typing to perform repetitive tasks, unaware that the machine they’re using could do their job in seconds if they gave it the right instructions'' \cite{Swei15}.  For my thesis project of the GymInf formation, I intend to build a suite of tools to support a range of teacher tasks including 
\begin{itemize} 
\item organizing seating plans \\
\emph{to aid the atmosphere and relationships} 
\item suggesting teacher-student interactions for upcoming classes \\
\emph{to foster relationships}
\item building individual student reports \\
\emph{to regulate cadence and scaffolding, as well as for follow-ups}
\item creating spreadsheets of marks \\
\emph{for follow-ups}
\end{itemize}

My plan is to 


\section{Introduction}

Bucheton and Soulé have descriped the act of teaching as a multiagenda game of postures requiring good preparation and excellent micro-decisions \cite{BS09}. In their model, they identify the crucial roles of the teacher in controling the cadence, the atmosphere, the scaffolding and the relationships during the class, and how each supports the learning objective. And yet many of the teacher's most time'consuming tasks do not take place in the classroom: good preparation and strong follow-ups (auto-reflection, marking, parent-teacher interactions) work to support and complement the overall success of the student learning process.

For example, teacher responsibilities include the ability
\begin{itemize}
\item To plan and implement effective classroom management practices
\item To design and implement effective strategies to develop independent learners
\item To engage students in active, hands-on, creative problem-based learning
\item To build students’ ability to work collaboratively with others
\item To maintain a safe, orderly environment conducive to learning
\item To adapt instruction/support to students’ differences in development, learning styles, strengths and needs
\item To write student reports to guide changes in instruction and practice, and to improve student learning
\end{itemize}

Many of these tasks are ripe for automation \cite{Swei15}. I would argue that some of these tasks should \textbf{not} be automated. As John Hattie explains in \emph{Visible Learning} \cite{Hat12}, \emph{``Expert teachers monitor learning and provide feedback.''} In my opinion writing student reports are a perfect example of a necessary evil: although time consuming (and potentially stressful) for the teacher, writing a student report forces the teacher to reflect on the progress of the student and at the same time manage the expectations of all partners - student, teacher, management and parent.

For the Master's thesis project undertaken for the GymInf formation, I intend to build a suite of tools to support a range of teacher tasks including organising seating plans, building individual student reports, suggesting teacher-student interactions for upcoming classes, and creating spreadsheets of marks.

\section{System plan}

\subsection{InOut}

We can analyse a system by connecting its (physical or virtual) inputs and outputs. In this project, we have

\

\begin{minipage}[t]{0.45\textwidth}

\textbf{Inputs}

\begin{itemize}
\item Class lists
\item Commentary on students \\
(during class)
\item Exam notes
\end{itemize}


\end{minipage}
\hfill
\vline
\hfill
\begin{minipage}[t]{0.45\textwidth}

\textbf{Outputs}

\begin{itemize}
\item Seating plans
\item Suggested students for focus \\
(during planning)
\item Average notes for the semester
\item Individual student reports
\end{itemize}



\end{minipage}

\section{Choice of programming language}

\section{Test driven development}

\section{Useful quotes}
 
\begin{center} 
\emph{``The best climate for learning is one in which there is trust. Students often don’t like to make mistakes because they fear a negative response from peers. Expert teachers create classrooms in which errors are welcome and learning is cool.''} \cite{Hat12}
\end{center}
 
\begin{center} 
\emph{``Since it has been reasonably well established that student affect toward a class is related to student learning, student attitudes toward classroom arrangements are a matter of no small concern when determining a choice of classroom arrangement.''} \cite{MM78}
\end{center}

\begin{thebibliography}{2}

\bibitem{Beck03} Beck, K. (2003). Test driven development by example. Addison-Wesley.

\bibitem{Bloom79} Bloom, B. S. (1976). Human characteristics and school learning. McGraw-Hill.

\bibitem{BS09} Bucheton D. \& Soulé Y. (2009). Les gestes professionnels et le jeu des postures de l’enseignant dans la classe : un multi-agenda de préoccupations enchâssées.

\bibitem{Hat12} Hattie, J. (2012). Visible learning for teachers: Maximizing impact on learning. Routledge.

\bibitem{MM78} McCorskey, J. C., \& McVetta, R. W. (1978). Classroom seating arrangements: Instructional communication theory versus student preferences. Communication education, 27(2), 99-111.

\bibitem{Swei15} Sweigart, A. (2015). Automate the Boring Stuff with Python.
\end{thebibliography}

\end{document}