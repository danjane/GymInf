%! Author = danjane
%! Date = 17.09.22

\documentclass[10pt]{article}

\usepackage{amsmath}
\usepackage{cmbright}

\begin{document}

\section{Project proposal}

Boucheton and Soulé have descriped the act of teaching as a multiagenda game of postures requiring good preparation and excellent micro-decisions \cite{BS09}. Many of the teacher's tasks are secondary to the main teacher-student interactions which take place in the classroom, and yet they work to support and complement the overall success of the student learning process. For example, teacher responsibilities include the ability
\begin{itemize}
\item To plan and implement effective classroom management practices
\item To design and implement effective strategies to develop independent learners
\item To engage students in active, hands-on, creative problem-based learning
\item To build students’ ability to work collaboratively with others
\item To maintain a safe, orderly environment conducive to learning
\item To adapt instruction/support to students’ differences in development, learning styles, strengths and needs
\item To write student reports to guide changes in instruction and practice, and to improve student learning
\end{itemize}
I would argue that many of these tasks should \textbf{not} be automated. As John Hattie explains in \emph{Visible Learning} \cite{Hat12}, \emph{``Expert teachers monitor learning and provide feedback.''} Writing student reports are a perfect example: although time consuming (and potentially stressful) for the teacher, writing a student report forces the teacher to reflect on the progress of the student and at the same time manage expectations on all sides - student, teacher and parent.

However, this still leaves many tasks which are ripe for automation \cite{Swei15}. For the Master's thesis project completed for the GymInf formation, I intend to build a suite of tools to automate (or at least provide support) for a range of teacher tasks including organising seating plans, building individual student reports, suggesting teacher-student interactions for upcoming classes, and creating spreadsheets of marks.

\section{System plan}

\subsection{InOut}

We can analyse a system by connecting its (physical or virtual) inputs and outputs. In this project, we have

\

\begin{minipage}[t]{0.45\textwidth}

\textbf{Inputs}

\begin{itemize}
\item Class lists
\item Commentary on students \\
(during class)
\item Exam notes
\end{itemize}


\end{minipage}
\hfill
\vline
\hfill
\begin{minipage}[t]{0.45\textwidth}

\textbf{Outputs}

\begin{itemize}
\item Seating plans
\item Suggested students for focus \\
(during planning)
\item Average notes for the semester
\item Individual student reports
\end{itemize}



\end{minipage}

\section{Choice of programming language}

\section{Test driven development}

\section{Useful quotes}
 
\begin{center} 
\emph{``The best climate for learning is one in which there is trust. Students often don’t like to make mistakes because they fear a negative response from peers. Expert teachers create classrooms in which errors are welcome and learning is cool.''} \cite{Hat12}
\end{center}
 
\begin{center} 
\emph{``Since it has been reasonably well established that student affect toward a class is related to student learning, student attitudes toward classroom arrangements are a matter of no small concern when determining a choice of classroom arrangement.''} \cite{MM78}
\end{center}

\begin{thebibliography}{2}

\bibitem{Bloom79} Bloom, B. S. (1976). Human characteristics and school learning. McGraw-Hill.

\bibitem{BS09} Bucheton D. \& Soulé Y. (2009). Les gestes professionnels et le jeu des postures de l’enseignant dans la classe : un multi-agenda de préoccupations enchâssées.

\bibitem{Hat12} Hattie, J. (2012). Visible learning for teachers: Maximizing impact on learning. Routledge.

\bibitem{MM78} McCorskey, J. C., \& McVetta, R. W. (1978). Classroom seating arrangements: Instructional communication theory versus student preferences. Communication education, 27(2), 99-111.

\bibitem{Swei15} Sweigart, A. (2015). Automate the Boring Stuff with Python.
\end{thebibliography}

\end{document}